\chapter{Conclusión}

A lo largo del informe se analizó el Teorema de la Función Implícita como herramienta para describir localmente curvas y superficies definidas por ecuaciones implícitas. Se vio que la existencia y unicidad de una solución dependen de una condición determinante: que la derivada parcial respecto de la variable a despejar no sea nula. Cuando esto ocurre, la ecuación puede escribirse como un gráfico suave, lo que habilita la diferenciación implícita y la aproximación lineal.\

Geométricamente, esta condición excluye configuraciones verticales: una curva de nivel puede expresarse como \(y=F(x)\) cuando el gradiente no es vertical en la dirección de \(y\), y una superficie \(f(x,y,z)=0\) puede escribirse como \(z=F(x,y)\) siempre que \(f_z\neq0\). La relación con el Teorema de la Función Inversa mostró, además, que el Teorema de la Función Implícita surge como un caso particular de la invertibilidad local de una transformación construida adecuadamente.\

Finalmente, el análisis del tercer ejercicio evidenció cómo estas ideas se extienden al estudio de la regularidad: al reescribir la relación \(g=H \circ f\) y utilizar la suavidad y la invertibilidad de \(H\), se concluyó que la suavidad infinita de \(g\) se transfiere directamente a \(f\). Esto ilustra el papel estructural de los teoremas de la función implícita e inversa en la teoría de la diferenciabilidad.\

También se discutieron sus límites: el resultado es estrictamente local y deja de aplicarse cuando la derivada relevante se anula. En esos puntos pueden aparecer pliegues, pérdida de suavidad o estructuras que no pueden describirse como gráficos; el teorema tampoco aporta información sobre la geometría global ni sobre el dominio máximo de la solución.\

Finalmente, el tercer ejercicio ilustró cómo la suavidad se preserva mediante composiciones con funciones auxiliares invertibles, reforzando la idea de que la invertibilidad, local o global, es el mecanismo que sostiene la regularidad en este tipo de problemas.