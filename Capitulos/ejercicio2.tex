\chapter{Ejercicio 2}

% ------------------------------------------------------------
% PARTE 1
% ------------------------------------------------------------

\subsection*{Determine si es posible expresar \(y\) como función de \(x\) cerca del punto \((0,1)\). En caso de serlo, calcule la derivada \(\frac{\partial y}{\partial x}\) en dicho punto.}
\addcontentsline{toc}{subsection}{Expresar \(y\) como función de \(x\) cerca del punto \((0,1)\)}

\[
\text{Considere la función}\quad f(x,y)=e^x+xy-2y^2
\]

Queremos ver si existe \(y=g(x)\) tal que
\[
f(x,g(x))=0
\]
Evaluar \((0,1)\)
\[
f(0,1)=e^0+0\cdot1-2(1)^2=-1 \neq 0
\]
El punto \((0,1)\) pertenece a la curva de nivel
\[
f(x,y)=f(0,1)=-1
\]
Usamos el teorema para \(f(x,y)=c\) fuera de la función original. El punto \(0,1\) no pertenece a la curva \(f(x,y)=0\), sino a la curva de nivel \(f(x,y)=-1\).
Para aplicar el Teorema de la Función Implícita basta trabajar con la ecuación que define la curva que pasa por el punto.\\
\(\exists \ y=F(x)\) clase \(C^1\) en un vecindario \(x=0\) siempre que:
\[
\begin{array}{c}
\frac{\partial f}{\partial y}(0,1)\neq 0 \\ [6pt]
\frac{\partial f}{\partial y}(x,y)=x-4y \to en (0,1) = -4 \neq 0
\end{array}
\]
Es posible expresar \(y\) como función de \(x\) localmente:
\[
\begin{array}{c}
y=F(x) \text{ tal que } f(x,F(x))=-1 \\ [6pt]
F'(x_0)=-\frac{f_x(x_0,y_0)}{f_y(x_0,y_0)} \\ [6pt]
f_x(x,y)=e^x+y, \quad f_y(x_0,y_0)=x-4y
\end{array}
\]
En \((0,1)\):
\[
\begin{array}{c}
f_x(0,1)=2, \quad f_y(0,1)=-4 \\ [6pt]
\frac{\partial y}{\partial x}(0,1) = -\frac{2}{-4} = \frac{1}{2}
\end{array}
\]
En el entorno del punto \((0,1)\), la ecuación \(f(x,y)=-1\) define de manera única una función \(y=F(x)\) de clase \(C^1\). La derivada de dicha función en el punto dado es:
\[
\frac{\partial y}{\partial x}(0,1) = \frac{1}{2}
\]

% ------------------------------------------------------------
% PARTE 2
% ------------------------------------------------------------

\subsection*{Explique cómo se relaciona el resultado anterior con las curvas de nivel de \(f(x,y)\).}
\addcontentsline{toc}{subsection}{Relación entre el resultado anterior y las curvas de nivel}

Curva de nivel relevante:
\[
f(x,y) = e^x+xy-2y^2=-1, \text{ porque } f(0,1)=-1
\]
Esa curva es el conjunto de puntos que mantienen constante el valor de f.\\
Cuando decimos que \(y=F(x)\) existe, significa que esa curva de nivel (curva en el plano) puede parametrizarse localmente como; \((x,F(x))\). La curva se puede ver como el gráfico de la función cerca del punto dado.\\
\(F'(0)\) es la pendiente de esa curva en el punto \((0,1)\). La pendiente de la tangente a la curva de nivel \(f=-1\) en \((0,1)\).\\
\underline{Geométricamente}
\[
f(x,F(x))=-1
\]
al diferenciarla:
\[
f_x+fyF'(x)=0 \Rightarrow F'(x) = -\frac{f_x}{f_y}
\]
\underline{La relación del gradiente}
\[
\begin{array}{c}
\nabla f(0,1) \perp \text{ curva de nivel } f=-1 \\ [6pt]
\nabla f(x,y)=(f_x,f_y)=(e^x+y,x-4y)
\end{array}
\]
En \((0,1)\):
\[
\nabla f(0,1)=(2,-4)
\]
Un vector director de la curva de nivel en ese punto debe ser perpendicular a ese gradiente.\\
Busco \((1,m)\):
\[
\begin{array}{c}
(1,m)\cdot(2,-4)=0 \\ [6pt]
2-4m=0 \Rightarrow m=\frac{1}{2}
\end{array}
\]

% ------------------------------------------------------------
% PARTE 3
% ------------------------------------------------------------


\subsection*{Realice una aproximación lineal del valor de \(y\) cerca de \((0,1)\). Estime el valor cuando \(x=0,1\).}
\addcontentsline{toc}{subsection}{Aproximación Lineal cuando \(x=0,1\).}

Tenemos, \(y(0)=1, \quad y'(0)=1/2\). La aproximación lineal sería:
\[
y(x) \approx 1+\frac{1}{2}x
\]
Entonces:
\[
y(0,1) \approx 1+\frac{1}{2}(0,1)=1+0,05=1,05
\]
Estimación: \(y(0,1)\approx 1,05\)

\begin{figure}[ht!] % Cambiar la h! cambia la posición de la imagen en el documento
    \centering
    \includegraphics[scale=0.3]{Imagenes/ex+xy-2y2puntos.png} % Acá va la ruta a la imagen
    \caption{Imagen de Geogebra con aproximación \(x=0\).} % Pie de imagen
    \label{fig:función2-top-view} % Esto sirve para referirnos a la imagen en el texto
    \includegraphics[scale=0.3]{Imagenes/ex+xy-2y2.png} % Acá va la ruta a la imagen
    \caption{Representación de la curva de nivel en \(x=0\)} % Pie de imagen
    \label{fig:función2-side-view}
\end{figure}

% ------------------------------------------------------------
% PARTE 4
% ------------------------------------------------------------

\clearpage
\subsection*{Suponga ahora que \(f(x,y,z)=e^x+xy-2y^2+z^3\). ¿Cómo se interpretaría el teorema en este contexto de tres variables, donde \(f(x,y,z)=0\) define una superficie?}
\addcontentsline{toc}{subsection}{Suponga ahora que \(f(x,y,z)=e^x+xy-2y^2+z^3\).}

Se quiere determinar si es posible describir dicha superficie \(f(x,y,z)=0\) localmente como el gráfico de una función
\[
z=F(x,y)
\]
cercano de un punto \((x_0,y_0,z_0)\) que satisfaga \(f(x_0,y_0,z_0)=0\).\\
El teorema establece que esto es posible siempre que
\[
f_z(x_0,y_0,z_0)\neq 0
\]

En este caso; \(f_z=3z^2\) es aplicable en cualquier punto con \(z_0\neq 0\). Existe una función: \(z=g(x,y)\), tal que localmente:
\[
e^x+xy-2y^2+(g(x,y))^3=0
\]
y sus derivadas:
\[
g_x(x_0,y_0)=-\frac{f_x}{f_z},\quad g_y(x_0,y_0)=-\frac{f_y}{f_z}
\]
\(f=0\) se puede ver como la gráfica de una función \(z=g(x,y)\). Es solo posible cuando la superposición no es vertical respecto a la dirección de \(z\). La condición \(f_z\neq 0\) implica que la superficie no presenta verticalidad en la dirección del eje \(z\): su normal no es paralela al plano \(xy\) \\
En cambio, en puntos donde \(z=0\), se tiene \(f_z=0\) y el Teorema de la función implícita no garantiza la existencia de una función \(z=F(x,y)\); la superficie podría tener un pliegue o cambiar de orientación.

