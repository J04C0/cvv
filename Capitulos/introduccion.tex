\chapter{Introducción}

En este informe se estudia el Teorema de la Función Implícita y su aplicación en diversos problemas de Cálculo en Varias Variables. El objetivo central es analizar cómo este teorema permite describir localmente curvas y superficies definidas por ecuaciones implícitas, garantizando la existencia, unicidad y suavidad de las funciones que las parametrizan.

Cada ejercicio aborda un aspecto específico del resultado: la verificación rigurosa de sus hipótesis, el cálculo de derivadas mediante diferenciación implícita, la interpretación geométrica de curvas de nivel y la extensión del teorema a funciones de tres variables. Además, se discute la relación entre el Teorema de la Función Implícita y el Teorema de la Función Inversa, mostrando cómo este último permite justificar la estructura local de las transformaciones que intervienen en la reformulación del problema.

A lo largo del trabajo se combinan argumentos analíticos con interpretaciones geométricas que ponen de manifiesto el papel del gradiente en la estructura de las curvas y superficies de nivel, proporcionando una visión integral del uso del teorema en distintos contextos.