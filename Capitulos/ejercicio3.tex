\chapter{Ejercicio 3}

\subsection*{Sean \(f,g: \mathbb{R}\to \mathbb{R}\) tales que \(g(x)=f(x)+(f(x))^5\quad \forall x \in \mathbb{R}\). Probar que si \(g\) es de clase \(C^\infty\), entonces \(f\) también lo es.}
\addcontentsline{toc}{subsection}{Relación g de \(C^\infty\) con \(f\)}

La función debe reescribirse introduciendo la función auxiliar:
\[
H:\mathbb{R} \to \mathbb{R}, \quad H(y)=y+y^5
\]
con esta notación se obtiene \(g(x)=H(f(x))\).\\
Por lo tanto, la relación \(g\) y \(f\) es una composición:
\[
g=H \circ f
\]
para estudiar la regularidad de \(f\), es escencial comprender las propiedades de \(H\), especialmente la regularidad de su inversa.
H es un polinomio, y por tanto pertenece a la clase \(C^\infty\).\\
Su derivada es:
\[
H'(y)=1+5y^4>0, \quad \forall y \in \mathbb{R}
\]
entonces:
\begin{itemize}
    \item \(H\) es inyectiva
    \item \(H\) es estrictamente creciente
    \item Al ser continua y con límite \(\pm\infty\) cuando \(y\to\pm\infty\), es sobreyectiva.
    \item \(H\) es biyectiva y posee una inversa global: \(H^{-1}:\mathbb{R}\to\mathbb{R}\)
\end{itemize}
Además por el teorema de función inversa dado que \(H'(y)\neq0\) en todo punto, la inversa \(H^{-1}\) no solo existe sino que es suave:
\[
H^{-1}\in C^\infty(\mathbb{R})
\]
A partir de la identidad \(g=H(f)\), se aplica \(H^{-1}\) a ambos lados: 
\[
f(x)=H^{-1}(g(x))
\]
caracteriza a \(f\) en términos de \(g\) y de \(H^{-1}\). Dado que \(g\in C^\infty\)(hipótesis) y \(H^{-1}\in C^\infty(\mathbb{R})\) por el análisis, se concluye que la función compuesta
\[
f=H^{-1}\circ g
\]
es también de clase \(C^\infty\), porque la composición de funciones suaves sigue siendo suave. La regularidad infinita de \(g\) se transmite a \(f\) sin pérdida alguna, gracias a la suavidad global de \(H^{-1}\).